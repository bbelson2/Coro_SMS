\documentclass[format=acmsmall, review=false, screen=false]{acmart}
%DIF LATEXDIFF DIFFERENCE FILE
%DIF DEL Coroutines-SMS-37-TECS.tex   Mon Oct  8 13:52:46 2018
%DIF ADD Coroutines-SMS-38-TECS.tex   Wed Feb  6 00:39:57 2019

%\citestyle{acmnumeric}
\citestyle{acmauthoryear}

\usepackage{booktabs} % For formal tables

\usepackage[ruled]{algorithm2e} % For algorithms
\renewcommand{\algorithmcfname}{ALGORITHM}
\SetAlFnt{\small}
\SetAlCapFnt{\small}
\SetAlCapNameFnt{\small}
\SetAlCapHSkip{0pt}
\IncMargin{-\parindent}

\usepackage{listings} % For code listings
%\lstdefinestyle{cppstyle}{
%	tabsize=2
%}
\definecolor{verylightgrey}{gray}{0.9}
\lstset{ %
	language=C++,
	basicstyle=\small,%
%	backgroundcolor=\color{verylightgrey}
	frame=single
	}

\usepackage{amssymb}
\usepackage[flushleft]{threeparttable}

\usepackage{graphicx}
\usepackage{subcaption}
\graphicspath{{images/}}

\usepackage{longtable}
\usepackage{multirow, makecell}
\usepackage{rotating}

% Metadata Information
\acmJournal{TECS}
%\acmVolume{1}
%\acmNumber{1}
%\acmArticle{1}
%\acmYear{2018}
%\acmMonth{1}
\copyrightyear{2018}
%\acmArticleSeq{1}

% Copyright
%\setcopyright{acmcopyright}
%\setcopyright{acmlicensed}
%\setcopyright{rightsretained}
%\setcopyright{usgov}
%\setcopyright{usgovmixed}
%\setcopyright{cagov}
%\setcopyright{cagovmixed}

% DOI
\acmDOI{}

% Paper history
%\received{February 2007}
%\received[revised]{March 2009}
%\received[accepted]{June 2009}


% Document starts
%DIF PREAMBLE EXTENSION ADDED BY LATEXDIFF
%DIF UNDERLINE PREAMBLE %DIF PREAMBLE
\RequirePackage[normalem]{ulem} %DIF PREAMBLE
\RequirePackage{color}\definecolor{RED}{rgb}{1,0,0}\definecolor{BLUE}{rgb}{0,0,1} %DIF PREAMBLE
\providecommand{\DIFadd}[1]{{\protect\color{blue}\uwave{#1}}} %DIF PREAMBLE
\providecommand{\DIFdel}[1]{{\protect\color{red}\sout{#1}}}                      %DIF PREAMBLE
%DIF SAFE PREAMBLE %DIF PREAMBLE
\providecommand{\DIFaddbegin}{} %DIF PREAMBLE
\providecommand{\DIFaddend}{} %DIF PREAMBLE
\providecommand{\DIFdelbegin}{} %DIF PREAMBLE
\providecommand{\DIFdelend}{} %DIF PREAMBLE
%DIF FLOATSAFE PREAMBLE %DIF PREAMBLE
\providecommand{\DIFaddFL}[1]{\DIFadd{#1}} %DIF PREAMBLE
\providecommand{\DIFdelFL}[1]{\DIFdel{#1}} %DIF PREAMBLE
\providecommand{\DIFaddbeginFL}{} %DIF PREAMBLE
\providecommand{\DIFaddendFL}{} %DIF PREAMBLE
\providecommand{\DIFdelbeginFL}{} %DIF PREAMBLE
\providecommand{\DIFdelendFL}{} %DIF PREAMBLE
\newcommand{\DIFscaledelfig}{0.5}
%DIF HIGHLIGHTGRAPHICS PREAMBLE %DIF PREAMBLE
\RequirePackage{settobox} %DIF PREAMBLE
\RequirePackage{letltxmacro} %DIF PREAMBLE
\newsavebox{\DIFdelgraphicsbox} %DIF PREAMBLE
\newlength{\DIFdelgraphicswidth} %DIF PREAMBLE
\newlength{\DIFdelgraphicsheight} %DIF PREAMBLE
% store original definition of \includegraphics %DIF PREAMBLE
\LetLtxMacro{\DIFOincludegraphics}{\includegraphics} %DIF PREAMBLE
\newcommand{\DIFaddincludegraphics}[2][]{{\color{blue}\fbox{\DIFOincludegraphics[#1]{#2}}}} %DIF PREAMBLE
\newcommand{\DIFdelincludegraphics}[2][]{% %DIF PREAMBLE
\sbox{\DIFdelgraphicsbox}{\DIFOincludegraphics[#1]{#2}}% %DIF PREAMBLE
\settoboxwidth{\DIFdelgraphicswidth}{\DIFdelgraphicsbox} %DIF PREAMBLE
\settoboxtotalheight{\DIFdelgraphicsheight}{\DIFdelgraphicsbox} %DIF PREAMBLE
\scalebox{\DIFscaledelfig}{% %DIF PREAMBLE
\parbox[b]{\DIFdelgraphicswidth}{\usebox{\DIFdelgraphicsbox}\\[-\baselineskip] \rule{\DIFdelgraphicswidth}{0em}}\llap{\resizebox{\DIFdelgraphicswidth}{\DIFdelgraphicsheight}{% %DIF PREAMBLE
\setlength{\unitlength}{\DIFdelgraphicswidth}% %DIF PREAMBLE
\begin{picture}(1,1)% %DIF PREAMBLE
\thicklines\linethickness{2pt} %DIF PREAMBLE
{\color[rgb]{1,0,0}\put(0,0){\framebox(1,1){}}}% %DIF PREAMBLE
{\color[rgb]{1,0,0}\put(0,0){\line( 1,1){1}}}% %DIF PREAMBLE
{\color[rgb]{1,0,0}\put(0,1){\line(1,-1){1}}}% %DIF PREAMBLE
\end{picture}% %DIF PREAMBLE
}\hspace*{3pt}}} %DIF PREAMBLE
} %DIF PREAMBLE
\LetLtxMacro{\DIFOaddbegin}{\DIFaddbegin} %DIF PREAMBLE
\LetLtxMacro{\DIFOaddend}{\DIFaddend} %DIF PREAMBLE
\LetLtxMacro{\DIFOdelbegin}{\DIFdelbegin} %DIF PREAMBLE
\LetLtxMacro{\DIFOdelend}{\DIFdelend} %DIF PREAMBLE
\DeclareRobustCommand{\DIFaddbegin}{\DIFOaddbegin \let\includegraphics\DIFaddincludegraphics} %DIF PREAMBLE
\DeclareRobustCommand{\DIFaddend}{\DIFOaddend \let\includegraphics\DIFOincludegraphics} %DIF PREAMBLE
\DeclareRobustCommand{\DIFdelbegin}{\DIFOdelbegin \let\includegraphics\DIFdelincludegraphics} %DIF PREAMBLE
\DeclareRobustCommand{\DIFdelend}{\DIFOaddend \let\includegraphics\DIFOincludegraphics} %DIF PREAMBLE
\LetLtxMacro{\DIFOaddbeginFL}{\DIFaddbeginFL} %DIF PREAMBLE
\LetLtxMacro{\DIFOaddendFL}{\DIFaddendFL} %DIF PREAMBLE
\LetLtxMacro{\DIFOdelbeginFL}{\DIFdelbeginFL} %DIF PREAMBLE
\LetLtxMacro{\DIFOdelendFL}{\DIFdelendFL} %DIF PREAMBLE
\DeclareRobustCommand{\DIFaddbeginFL}{\DIFOaddbeginFL \let\includegraphics\DIFaddincludegraphics} %DIF PREAMBLE
\DeclareRobustCommand{\DIFaddendFL}{\DIFOaddendFL \let\includegraphics\DIFOincludegraphics} %DIF PREAMBLE
\DeclareRobustCommand{\DIFdelbeginFL}{\DIFOdelbeginFL \let\includegraphics\DIFdelincludegraphics} %DIF PREAMBLE
\DeclareRobustCommand{\DIFdelendFL}{\DIFOaddendFL \let\includegraphics\DIFOincludegraphics} %DIF PREAMBLE
%DIF LISTINGS PREAMBLE %DIF PREAMBLE
\lstdefinelanguage{codediff}{ %DIF PREAMBLE
  moredelim=**[is][\color{red}]{*!----}{----!*}, %DIF PREAMBLE
  moredelim=**[is][\color{blue}]{*!++++}{++++!*} %DIF PREAMBLE
} %DIF PREAMBLE
\lstdefinestyle{codediff}{ %DIF PREAMBLE
	belowcaptionskip=.25\baselineskip, %DIF PREAMBLE
	language=codediff, %DIF PREAMBLE
	basicstyle=\ttfamily, %DIF PREAMBLE
	columns=fullflexible, %DIF PREAMBLE
	keepspaces=true, %DIF PREAMBLE
} %DIF PREAMBLE
%DIF END PREAMBLE EXTENSION ADDED BY LATEXDIFF

\begin{document}
% Title portion. Note the short title for running heads
\DIFdelbegin %DIFDELCMD < \title[A survey of asynchronous programming using coroutines in IoT and embedded systems]{%%%
\DIFdelend \DIFaddbegin \title[A Survey of Asynchronous Programming using Coroutines in IoT and Embedded Systems]{\DIFaddend A \DIFdelbegin \DIFdel{survey }\DIFdelend \DIFaddbegin \DIFadd{Survey }\DIFaddend of \DIFdelbegin \DIFdel{asynchronous programming }\DIFdelend \DIFaddbegin \DIFadd{Asynchronous Programming }\DIFaddend using \DIFdelbegin \DIFdel{coroutines }\DIFdelend \DIFaddbegin \DIFadd{Coroutines }\DIFaddend in the Internet of Things and \DIFdelbegin \DIFdel{embedded systems}\DIFdelend \DIFaddbegin \DIFadd{Embedded Systems}\DIFaddend }

\author{Bruce Belson}
\orcid{0000-0003-2904-1650}
\affiliation{%
  \institution{James Cook University}
  \department{College of Science \& Engineering}
  \city{Cairns}
  \state{Queensland}
  \postcode{4870}
  \country{Australia}}
\email{bruce.belson@my.jcu.edu.au}

\author{Jason Holdsworth}
\affiliation{%
	\institution{James Cook University}
	\department{College of Business, Law \& Governance}
	%\city{Cairns}
	%\state{Queensland}
	%\postcode{4870}
	%\country{Australia}
}
\email{jason.holdsworth@jcu.edu.au}

\author{Wei Xiang}
\affiliation{%
	\institution{James Cook University}
	\department{College of Science \& Engineering}
	%\city{Cairns}
	%\state{Queensland}
	%\postcode{4870}
	%\country{Australia}
}
\email{wei.xiang@jcu.edu.au}

\author{Bronson Philippa}
\affiliation{%
	\institution{James Cook University}
	\department{College of Science \& Engineering}
	%\city{Cairns}
	%\state{Queensland}
	%\postcode{4870}
	%\country{Australia}
}
\email{bronson.philippa@jcu.edu.au}



\begin{abstract}
Many Internet of Things and embedded projects are event-driven, and therefore require asynchronous and concurrent programming. Current proposals for C++2020 suggest that coroutines will have native language support. It is timely to survey the current use of coroutines in embedded systems development.  This paper investigates existing research which uses or describes coroutines on resource-constrained platforms. The existing research is analysed with regard to: software platform, hardware platform and capacity; use cases and intended benefits; and the application programming interface design used for coroutines. A systematic mapping study was performed, to select studies published between 2007 and 2018 which contained original research into the application of coroutines on resource-constrained platforms. An initial set of 566 candidate papers, collated from on-line databases, were reduced to only 35 after filters were applied, revealing the following taxonomy. The C \& C++ programming languages were used by 22 studies out of 35. As regards hardware, 16 studies used 8- or 16-bit processors while 13 used 32-bit processors.  The four most common use cases were concurrency (17 papers), network communication (15), sensor readings (9) and data flow (7). The leading intended benefits were code style and simplicity (12 papers), scheduling (9) and efficiency (8). A wide variety of techniques have been used to implement coroutines, including native macros, additional tool-chain steps, new language features and non-portable assembly language. We conclude that there is widespread demand for coroutines on resource-constrained devices. Our findings suggest that there is significant demand for a formalised, stable, well-supported implementation of coroutines in C++, designed with consideration of the special needs of resource-constrained devices, and further that such an implementation would bring benefits specific to such devices.
\end{abstract}


%
% The code below should be generated by the tool at
% http://dl.acm.org/ccs.cfm
% Please copy and paste the code instead of the example below.
%
\begin{CCSXML}
	<ccs2012>
	<concept>
	<concept_id>10011007.10011006.10011008.10011024.10011037</concept_id>
	<concept_desc>Software and its engineering~Coroutines</concept_desc>
	<concept_significance>500</concept_significance>
	</concept>
	<concept>
	<concept_id>10011007.10011006.10011041</concept_id>
	<concept_desc>Software and its engineering~Compilers</concept_desc>
	<concept_significance>300</concept_significance>
	</concept>
	<concept>
	<concept_id>10010520.10010553.10010562.10010564</concept_id>
	<concept_desc>Computer systems organization~Embedded software</concept_desc>
	<concept_significance>300</concept_significance>
	</concept>
	<concept>
	<concept_id>10010520.10010553.10010559</concept_id>
	<concept_desc>Computer systems organization~Sensors and actuators</concept_desc>
	<concept_significance>100</concept_significance>
	</concept>
	</ccs2012>
\end{CCSXML}

\ccsdesc[500]{Software and its engineering~Coroutines}
\ccsdesc[300]{Software and its engineering~Compilers}
\ccsdesc[300]{Computer systems organization~Embedded software}
\ccsdesc[100]{Computer systems organization~Sensors and actuators}


%
% End generated code
%


\keywords{embedded, resource-constrained, asynchronous, direct style, scheduling}

\maketitle

% If the default list of authors is too long for headers.
%\renewcommand{\shortauthors}{B. Belson et al.}

\DIFdelbegin %DIFDELCMD < \input{Coroutines-SMS-37}
%DIFDELCMD < %%%
\DIFdelend \DIFaddbegin \input{Coroutines-SMS-38}
\DIFaddend 

\DIFdelbegin %DIFDELCMD < % Appendix
\appendix

\section{Supplementary materials}
\begin{printonly}
	Supplementary materials are available in the online version of this paper.
\end{printonly}

\begin{screenonly}

\begin{table}[h]
	\caption{Research questions}
	
	\begin{tabular}{l l l}
\hline
Code & & Research question \\
\hline
RQ1	& & What was the software platform?\\
& RQ1a & What was the programming language used?\\
& RQ1b & What method was used to implement coroutines?\\
& RQ1c & What was the operating system used (if any)?\\
RQ2	& & What was the hardware platform?\\
& RQ2a & What was the class of hardware platform?\\
& RQ2b & How much read-only or flash memory (ROM) was available?\\
& RQ2c & How much random-access memory (RAM) was available?\\
& RQ2d & What was the processor family?\\
& RQ2e & Was the processor 8-bit, 16-bit or 32-bit?\\
& RQ2f & What was the processor’s instruction set?\\
RQ3	 &  & What were the use cases?\\
RQ4	 &  & What were the intended benefits of using coroutines in this context?\\
RQ5	 &  & What is the API of the coroutine?\\
& RQ5a & Does the paper discuss an implementation of coroutines?\\
& RQ5b & Is the control flow managed on behalf of the developer?\\
& RQ5c & Is the state of local variables automatically managed?\\
& RQ5d & Is the coroutine implementation stackless or stackful?\\
& RQ5e & How is the coroutine state allocated?\\
\hline
	\end{tabular}
\end{table}
	
\begin{table}[h]
	\caption{On-line databases of digital libraries}
	
	\begin{tabular}{l l}
		\hline
		Library	& URL \\
		\hline
		ACM Digital Library & http://dl.acm.org \\
		IEEE Xplore & http://ieeexplore.ieee.org \\
		ScienceDirect & http://www.sciencedirect.com/ \\
		Scopus & http://www.scopus.com/ \\
		SpringerLink & https://link.springer.com/ \\
		Web of Science & http://www.webofknowledge.com \\
		\hline
	\end{tabular}
\end{table}
	
\begin{table}[h]
	\caption{Inclusion criteria}
	
	\begin{tabular} {l p{10cm} }
		\hline
		Identifier & Criterion \\
		\hline
		IC1 & The paper contains original research into the application of coroutines on resource-constrained platforms.\\
		IC1a & The application of coroutines must extend to the code on the platform itself, not merely to the simulator of the platform.\\
		\hline
	\end{tabular}
\end{table}

\begin{table}[h]
	\caption{Exclusion criteria}
	
	\begin{tabular} {l p{10cm} }
		\hline
		Identifier & Criterion \\
		\hline
		EC1 & The paper has no digital object identifier (DOI) or International Standard Book Number (ISBN).\\
		EC2 & The paper has no abstract.\\
		EC3 & The paper was published before 2007.\\
		EC4 & The paper is not written in English.\\
		EC5 & The complete paper was not available to the reviewers in any form equivalent to the final version.\\
		EC6 & The paper is an earlier version of another candidate paper. \\
		EC7 & The paper is not a primary study. \\
		EC8 & The paper does not fall into any of the selected publication classes. \\
		\hline
	\end{tabular}
\end{table}
\end{screenonly}

%DIFDELCMD < %%%
\DIFdelend \DIFaddbegin \input{Coroutines-SMS-Supp-38}
\DIFaddend 


\end{document}
